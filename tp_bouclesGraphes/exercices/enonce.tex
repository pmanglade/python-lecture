%%\section{Exemples}


En calcul scientifique il est fréquent de travailler avec un grand nombre
(plusieurs milliards par exemple) de variables. Il est alors impossible de
leur donner des noms individuels, puis
de les manipuler avec des instructions utilisant de tels noms.

En Python les outils de base pour gérer de grand nombre de variable
sont les listes (\verb+list+), et les boucles \verb+for+ ou \verb+while+.

Et comme il convient de toujours vérifier les résultats que fourni un code, et
que vérifier une à une les valeurs d'un grand nombre de variables est
souvent délicat, cette section servira également d'introduction au tracé de
graphiques avec Python.

\subsection{Manipulation des listes}
Cet exercice consiste à parcourir des objets de type \texttt{list}.

\begin{enumerate}
\item Trouvez dans le code du programme \texttt{exo\_list.py} un objet
  de type liste.
\item Testez la fonction \verb+len()+ sur un tel objet. Que fait-elle ?
\item Soit \verb+L+ un objet de type liste, testez ce que retourne
  les syntaxes \verb+L[0]+, \verb+L[1]+, \verb+L[len(L)-1]+, \verb+L[len(L)]]+. 
\item  Dans le programme, faites  les
  modifications  nécessaires pour  afficher  les premier et  dernier
  éléments de l'objet \texttt{my\_list\_of\_heroes}.  %%
\item  Créez deux  \texttt{list} afin  de séparer  les héros  dotés de
  pouvoirs de celui doté d'un compte en banque.  %%
\item  Utilisez  la liste  vide  \texttt{[]}  pour \og{}vider\fg\  la  variable
  \texttt{my\_list\_of\_heroes},
  puis  vérifiez que  cette liste
  est effectivement vide.  %%
\item Que fait la commande de la ligne 20 ?  %%
\item  Afin de  déterminer la  position  d'un élément  connu dans  une
  liste, utilisez la commande \texttt{index(valeur)}.  Retrouvez ainsi
  la position de la chaîne \textit{"Daredevil"}.

\item En vous aidant  du programme \texttt{exo\_list.py}, expliquez la
  différence entre \texttt{del} et \texttt{remove}

\item Jusqu'à maintenant l'objet de type \texttt{list} utilisé compte quelques
  éléments. Dès lors que vous  devrez gérer un grand nombre d'éléments,
  il faudra  automatiser le parcours  de vos ensembles.   Décrivez les
  lignes 25 et 26 du programme.
\end{enumerate}
\begin{figure}  
  \lstinputlisting{../exemples/exo_list.py}
  \caption{Contenu du fichier \texttt{exo\_list.py}}
\end{figure}

\subsection{Les boucles  \texttt{for}}

L'objectif  de cet  exercice est  d'utiliser les  boucles basées sur
l'instruction \texttt{for} afin  de  parcourir  des  ensembles  de
valeurs  contenues  dans  des listes.

\begin{enumerate}
\item Vous  étudierez le programme \texttt{boucle\_for.py}.   Les mots
  clefs \texttt{for} et \texttt{in} séparent \texttt{heros}, qui prend
  successivement    les    différentes    valeurs    contenues    dans
  \texttt{my\_list\_of\_heroes}.  Détaillez la  différence entre les
  boucles commençant aux lignes 20 et 30.

\item La boucle que vous avez  utilisée jusqu'à présent se base sur une
  liste déjà existante.  Toutefois,  il est utile d'effectuer $N$
  fois une  action. Pour cela, à  la place d'une liste  de $N$
  éléments, vous utiliserez  l'instruction \texttt{range(N)} pour créer
  une liste de $N$ entiers correspondant à la suite des indices des éléments
  de la liste.
  \begin{itemize}
  \item[$\ast$]  Créez un  nouveau  programme \texttt{boucle\_for\_2.py}  pour
    mettre en \oe{}uvre cette boucle.
  \item Testez et expliquez les instructions suivantes.
\begin{verbatim}
N = 6
j=0
L=[]
for i in range(6):
  j=j+i
  L.append(j)
\end{verbatim}

  \item[$\ast$] Calculez la factorielle de \texttt{N}.
    Rappel : factorielle de
      $N$, notée $N!=1 \times 2 \times... \times N-1 \times N = (N-1)! \times N $ et $1! = 0! = 1$.
    \item Calculez les 10 premiers termes de la suite de Fibonacci sachant que $F_0=0$, $F_1=1$, et $F_n =F_{n-1}+F_{n-2}$. 
  \end{itemize}
\begin{figure}  
  \lstinputlisting{../exemples/boucle_for.py}
  \caption{Contenu du fichier \texttt{boucle\_for.py}}
\end{figure}

\item    Créez   un    nouveau   programme    que   vous    appellerez
  \texttt{boucle\_for\_3.py}.
  \begin{itemize}
  \item[$\ast$]  Demandez   à  un  utilisateur  d'entrer   un  entier  positif
    (\texttt{N} dans le programme).
  \item[$\ast$]  Proposez à  l'utilisateur de lui  produire des  nombres pairs  ou
    impairs, selon son choix.
  \item[$\ast$] Pour chaque valeur \texttt{x} de l'ensemble des entiers de 0
    à \texttt{N} calculez soit la  valeur $2\times x+1$ pour produire
    des nombres impairs, soit $2\times x$ pour des nombres pairs.
  \item[$\ast$]    Sauvegardez     votre    résultat    dans     un    fichier.
  \end{itemize}

\end{enumerate}




\subsection{Les boucles  \texttt{while}}

L'objectif de  cet exercice  est d'utiliser une  boucle basée sur l'instruction
\texttt{while}
afin  de  répéter  une  action tant  qu'une  condition demeure  vérifiée.
Inspirez-vous de  l'exemple \texttt{boucle\_while.py}  pour créer  un
programme.

\begin{enumerate}
\item Demandez à  un utilisateur de trouver un  nombre mystère compris
  entre 0 et 1000  tant qu'il ne trouve pas la  valeur exacte qui sera
  définie arbitrairement dans le programme.

\item  Après  chaque essai,  comparez  le  nombre proposé  au  nombre
  mystère et informez l'utilisateur s'il est plus grand ou plus petit.

\item Lorsque la valeur mystère est trouvée, informez l'utilisateur et
  stoppez le programme.

\end{enumerate}
%%

\begin{figure}  
  \lstinputlisting{../exemples/boucle_while.py}
  \caption{Contenu du fichier \texttt{boucle\_while.py}}
\end{figure}


\subsection{Vos premiers graphiques avec matplotlib}

Vous  trouverez   dans  le   programme  \texttt{plot.py}   un  exemple
d'utilisation du  module \texttt{matplotlib}. À partir  de cet exemple,
créez  un programme  pour  tracer  la fonction  $f(x)=\sin  x$ avec  x
compris entre 0 et 7.

Une liste de codes simples pour modifier l'affichage des courbes
à l'adresse :\\ \url{https://matplotlib.org/2.1.1/api/_as_gen/matplotlib.pyplot.plot.html}.
Pour vous exercer, modifiez les paramètres suivants :

\begin{enumerate}
\item les titres et les échelles des axes ;

\item la forme des  symboles représentant les points à tracer ;

\item reliez les points en utilisant une option de \texttt{plot()} ;

\item ajoutez une seconde courbe avant l'affichage représentant $f(x)=cos(x)$ et changez la couleur
  de la seconde courbe.


\end{enumerate}
\begin{figure}  
  \lstinputlisting{../exemples/plot.py}
  \caption{Contenu du fichier \texttt{plot.py}}
\end{figure}
% \subsection{Pour aller plus loin avec matplotlib {\sc [Facultatif]}}

% Vous aurez à écrire un programme afin  de tracer la carte de France et
% ses principaux cours d'eau.  Vous allez :

% \begin{enumerate}
% \item Lire les fichiers \texttt{map.dat} et \texttt{river.dat}. Chaque
%   colonne  de  ces fichiers  permet  de  remplir  des objets  de  type
%   \texttt{list} que  nous nommerons  \texttt{map\_x}, \texttt{map\_y},
%   \texttt{river\_x} et \texttt{river\_y}.
% \item Choisir la forme des symboles représentant les points à tracer.

% \item Sur un même graphique,  tracer les contours $map\_y = f(map\_x)$
%   et $river\_y = f(river\_x)$.

% \item Utiliser les options adéquates  pour tracer en noir les contours
%   de la carte et en bleu les cours d'eau.


% \end{enumerate}




\vfill
