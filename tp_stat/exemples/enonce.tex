%%\section{Exemples}

\subsection{Écrire dans un fichier}

Pour s'exercer à écrire dans un fichier, vous étudierez le fichier \texttt{write\_file.py}.

\begin{enumerate}

\item La ligne 21 permet d'ouvrir le fichier \texttt{quote\_file.dat} en écriture grâce
  à l'option \texttt{w}.
  Ajoutez à la fin du programme les lignes nécessaires afin d'ajouter, à la fin du
  fichier \texttt{quote\_file.dat}, la traduction de la citation.

\item   Écrivez   un  nouveau   programme   \texttt{write\_file\_2.py}
  permettant d'ouvrir  de nouveau le  fichier \texttt{quote\_file.dat}
  afin d'y ajouter une nouvelle citation.
    
\end{enumerate}

\subsection{Lire un fichier texte}

Pour  s'exercer à  lire dans  un  fichier, vous  étudierez le  fichier
\texttt{read\_file.py}.

\begin{enumerate}

\item Expliquez  les différences  entre les  méthodes \texttt{read()},
  \texttt{readlines()} et \texttt{readline()},

\item  Comment peut  on lire  l'ensemble  du fichier  avec la  méthode
  \texttt{readline()} ?

\item  Que se  passe-t-il lorsque  l'on \textit{oublie}  de fermer  le
  fichier à la fin du programme ?

\item Testez le type auquel appartient la variable \texttt{lines}.

\item Lors de l'affichage de  la variable \texttt{lines}, chaque ligne
  est encadré par un \texttt{'}  et les différentes lignes se trouvent
  séparées par une  virgule. L'ensemble est délimité  par des crochets
  \texttt{[]}.  Chaque  élément séparé par une  virgule est accessible
  indépendamment.  Cet objet est de  type \texttt{list}, c'est un type
  de  variable   qui  peut  lui-même  contenir   plusieurs  variables.
  Commentez  les lignes  49 et  50  puis proposez  deux méthodes  pour
  accéder à l'ensemble des éléments de l'objet \texttt{lines}

\end{enumerate}

\subsection{Manipulation des list}


Cet exercice consiste à parcourir des objets de type \texttt{list}.

\begin{enumerate}
\item  Ouvrez   le  programme  \texttt{exo\_list.py}  et   faites  les
  modifications  nécessaire pour  afficher  le premier  et le  dernier
  élément de l'objet \texttt{my\_list\_of\_heros}.  %%
\item  Créez deux  \texttt{list} afin  de séparer  les héros  dotés de
  pouvoir de celui doté d'un compte en banque.  %%
\item  Utilisez  la liste  vide  \texttt{[]}  pour vider  la  variable
  \texttt{my\_list\_of\_heros} puis  vérifiez que  cette \texttt{list}
  est effectivement vide.  %%
\item Que fait la commande de la ligne 21 ?  %%
\item  Afin de  déterminer la  position  d'un élément  connu dans  une
  \texttt{list}, utilisez la commande index(élément).  Retrouvez ainsi
  la position de \textit{Daredevil}.

\item En vous aidant  du programme \texttt{exo\_list.py}, expliquez la
  différence entre \texttt{del} et \texttt{remove}

\item Jusqu'à maintenant l'objet \texttt{list} utilisé compte quelques
  éléments. Dès lors que vous  devrez gérer un grand nombre d'élément,
  il faudra  automatiser le parcours  de vos ensembles.   Décrivez les
  lignes 33 et 34 de votre programme.
\end{enumerate}



\subsection{Les boucles \texttt{for}}

L'objectif  de cet  exercice est  d'utiliser les  boucles \texttt{for}
afin  de  parcourir  des  ensembles  de  valeurs  contenues  dans  des
\texttt{list}.



\begin{enumerate}
\item Vous  étudierez le programme \texttt{boucle\_for.py}.   Les mots
  clefs \texttt{for} et \texttt{in} séparent \texttt{heros}, qui prend
  successivement    les    différentes    valeurs    contenues    dans
  \texttt{my\_list\_of\_character}.  Détaillez la  différence entre la
  boucle commençant ligne 22 et la boucle commençant en ligne 35.

\item La boucle que vous avez  utilisé jusqu'à présent se base sur une
  \texttt{list} déjà existant.  Toutefois,  il est utile d'effectuer N
  fois une  action. Pour cela, à  la place d'une \texttt{list}  de $N$
  éléments, vous utiliserez  l'instruction \texttt{range(N)}pour créer
  une liste de $N$ entiers.
  \begin{itemize}
  \item[$\ast$]  Créez un  nouveau  programme \texttt{boucle\_for\_2.py}  pour
    mettre en oeuvre cette boucle.
  \item[$\ast$] Calculez la factoriellle de \texttt{N}\footnote{factorielle de
    N ou noté encore N!~=~$1 \times 2 \times... \times N-1 \times N$}.
  \end{itemize}


  
\item    Créez   un    nouveau   programme    que   vous    appellerez
  \texttt{boucle\_for\_3.py}.
  \begin{itemize}
  \item[$\ast$]  Demandez   à  un  utilisateur  d'entrer   un  entier  positif
    \texttt{N}.
  \item[$\ast$]  Proposez à  l'utilisateur de  produire des  nombres pairs  ou
    impairs.
  \item[$\ast$] Pour chaque valeur \texttt{x},  de l'ensemble des entiers de 0
    à \texttt{N}, calculez soit la  valeur $2\times x+1$ pour produire
    des nombres impairs et $2\times x$ pour des nombres pairs.
  \item[$\ast$]    Sauvegardez     votre    résultat    dans     un    fichier
    \texttt{ASCII}\footnote{American  Standard  Code  for  Information
      Interchange}.
  \end{itemize}

\end{enumerate}




\subsection{Les boucles \texttt{while}}

L'objectif de  cet exercice  est d'utiliser une  boucle \texttt{while}
afin  de  répéter  une  action tant  qu'une  condition  est  vérifiée.
Inspirez  vous de  l'exemple \texttt{boucle\_while.py}  pour créer  un
programme.

\begin{enumerate}
\item Demandez à  un utilisateur de trouver un  nombre mystère compris
  entre 0 et 1000  tant qu'il ne trouve pas la  valeur exacte qui sera
  définie dans le programme.

\item  Après  chaque essais,  comparez  le  nombre proposé  au  nombre
  mystère et dites s'il est plus grand ou plus petit.

\item Lorsque la valeur mystère est trouvée, informez l'utilisateur et
  stoppez le programme

\end{enumerate}
%%



\subsection{Vos premiers graphiques avec matplotlib}

Vous  trouverez   dans  le   programme  \texttt{plot.py}   un  exemple
d'utilisation du  module \texttt{matplotlib} À partir  de cet exemple,
créez  un programme  pour  tracer  la fonction  $f(x)=\sin  x$ avec  x
compris entre 0 et 7.

Pour vous exercer, modifiez les paramètres suivants :

\begin{enumerate}
\item les titres et les échelles des axes

\item la forme des  symboles représentant les points à tracer

\item reliez les points en utilisant une option de \texttt{plot()}

\item ajoutez une seconde courbe avant l'affichage représentant $f(x)=cos(x)$ et changez la couleur
  de la seconde courbe.


\end{enumerate}

\subsection{Pour aller plus loin avec matplotlib [OPTION]}

Vous aurez à écrire un programme afin de tracer la carte de france et ses principaux cours d'eau.
Vous allez : 

\begin{enumerate}
\item Lire les fichiers \texttt{carte.dat} et \texttt{river.dat}
\item Chaque colonne de ces fichiers permetrra de remplir des objets \texttt{list} que nous nommerons
  \texttt{carte\_x},  \texttt{carte\_y},  \texttt{river\_x} et \texttt{river\_y}.
\item la forme des  symboles représentant les points à tracer

\item Sur le même graphique, vous tracerez $carte\_y = f(carte\_x)$ et $river\_y = f(river\_x)$.

\item Utilisez les options adéquat pour tracer en noir les contours de la carte et en rouge les cours d'eau.


\end{enumerate}




\vfill
