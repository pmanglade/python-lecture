%%\section{Pair ou impair}

%%\section{Exercices}


\subsection{Construire un histogramme}

Dans cet exercice, à partir d'une  série de points que vous lirez d'un
fichier texte, vous construirez un histogramme représentant une loi de
distribution statistique.  Typiquement,  un histogramme  vous permettra
d'étudier  la   répartition  statistique  d'un  ensemble   de  valeurs
numériques.

Pour ce  faire, il est  nécessaire de  définir trois paramètres  : les
valeurs minimale et maximale de la gamme des données à étudier et le
nombre de classes de l'histogramme.
\begin{enumerate}
\item Demandez à un utilisateur d'entrer les valeurs min, max et nombre de classes ($min$, $max$ et $nb\_bin$).
% Pourquoi nb_bin est au singulier quand histo_bins est au pluriel ?
\item Créez une liste $histo\_bins$  contenant $nb\_bin$ cases dont les valeurs sont toutes égales à 0.
\item Calculez la largeur de répartition des classes ($bin\_width = (max - min) / nb\_bin$).
\item Lisez le fichier \texttt{data.dat} ligne à ligne.
\item Pour chaque valeur lue, trouvez à quelle classe appartient la valeur.
\item Lorsque la valeur appartient à la $i^{\`eme}$~classe, ajoutez $+1$ à la  $i^{eme}$~case de la liste  $histo\_bins$.
\item Tracez $histo\_bins$ en fonction de $i$ et comparez les résultats en fonction de la valeur  $nb\_bin$ entrée.
\end{enumerate}



\subsection{Statistiques élémentaires sur un fichier de données}

L'objectif de cet  exercice est d'étudier des données  fournies par la
station Météo-France de Caen-Carpiquet.
\begin{enumerate}
\item Avec un programme Python, ouvrez le fichier \texttt{temperature.dat}.

\item Utilisez une boucle \texttt{for} pour lire ligne à ligne le fichier de données.
  Puis affichez à l'écran la variable contenant la ligne courante.

\item De quel type est la variable contenant la ligne courante ?

%\item La fonctionalité \texttt{split(" ")} appliqué à une chaîne de caractères permet de
%  séparer cette chaîne de caractères en prenant l'option de \texttt{split} comme séparateur.
%  L'objet produit par cette action est une \texttt{list} de chaîne de caractères.

\item À chaque itération de la boucle, faites afficher la température et stockez chaque valeur dans une liste.

\item Sans écrire une ligne de programmation, proposez la logique pour trouver :
  \begin{itemize}
  \item[$\ast$] La température minimale
  \item[$\ast$] La température maximale
  \item[$\ast$] La température moyenne    $ \bar{T}= \frac{1}{n}  \sum\limits_{\substack{i=0}}^{n}{T_i}$
  \item[$\ast$] L'écart-type  $\sigma = \sqrt{\frac{1}{n}  \sum\limits_{\substack{i=0}}^{n}{} (T_i-\bar{T})^2   }$
  \end{itemize}

\item Réalisez ensuite ces calculs dans votre programme Python et enregistrez les résultats dans un fichier.

\item Tracez les valeurs des températures en utilisant le module \texttt{matplotlib}.

\item Faites ensuite le même travail pour la pression atmosphérique et pour l'humidité relative.

\item Sans écrire le programme, proposez une solution pour ranger dans l'ordre croissant
  les grandeurs que vous avez stockées dans des listes.\\
  \textit{OPTION : Implémentez l'algorithme de tri que vous avez formulé.}

\item Testez la fonction \texttt{.sort()} (méthode) sur un objet de type \texttt{list} et commentez.

\end{enumerate}
