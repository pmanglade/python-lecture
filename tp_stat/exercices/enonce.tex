%%\section{Pair ou impair}

\section{Exercices}

\subsection{Faire des statistiques élémentaires sur un fichier de données}

L'objectif de cet exercice est d'étudier les données fournies par la station météo-france
de Caen-Carpiquet.
\begin{enumerate}
\item Avec un programme python, ouvrez le fichier \texttt{temperature.dat}

\item Utilisez une boucle \texttt{for} pour lire ligne à ligne le fichier de données.
  Puis affichez à l'écran la variable contenant la ligne courante.

\item De quel type est la variable contenant la ligne courante ?

%\item La fonctionalité \texttt{split(" ")} appliqué à une chaîne de caractères permet de
%  séparer cette chaîne de caractères en prenant l'option de \texttt{split} comme séparateur.
%  L'objet produit par cette action est une \texttt{list} de chaîne de caractères.

\item À chaque itération de la boucle, faites afficher la température et stockez chaque valeur dans une \texttt{list} de températures.

\item Sans écrire une ligne de programmation , proposez la logique pour trouver :
  \begin{itemize}
  \item La température minimale
  \item La température maximale
  \item La température moyenne    $ \bar{T}= \frac{1}{n}  \sum\limits_{\substack{i=0}}^{n}{T_i}$
  \item L'écart-type  $\sigma = \sqrt{\frac{1}{n}  \sum\limits_{\substack{i=0}}^{n}{} (T_i-\bar{T})^2   }$
  \end{itemize}

\item Réalisez ensuite ces calculs dans votre programme python et enregistrez les résultats dans un fichier.

\item Tracez les valeurs des températures en utilisant le module \texttt{matplotlib}.

\item Vous ferez ensuite le même travail pour la pression atmosphérique et pour l'humidité relative.

\item Sans écrire le programme, proposez une solution pour ranger dans l'ordre croissant les grandeurs que vous avez stocké dans des \texttt{list}
  OPTION : Implémentez le tri que vous avez étudier.

\item Testez la fonctionnalité \texttt{.sort()} sur un objet de type \texttt{list} et commentez.

\end{enumerate}
