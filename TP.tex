%\documentclass[a4paper,12pt]{article}
%\documentclass[a4paper,landscape,11pt]{report}
\documentclass[a4paper,11pt]{article}
%
\usepackage{amsmath}
\usepackage{amssymb}
\usepackage{array}
\usepackage[french]{babel}
\usepackage{calc}
%\usepackage{caption}
\usepackage[hang]{caption}
\usepackage{cite}
\usepackage{color}
\usepackage{enumerate}
\usepackage{epic} 
\usepackage{eepic}
\usepackage{eepicemu}
\usepackage{epsfig}
\usepackage{fancyhdr}
\usepackage{float}
\usepackage[T1]{fontenc}
%\usepackage{fourier}
\usepackage[dvips]{geometry}
\usepackage{graphicx}
\usepackage{import} % To use \subimport and import stuff 
\usepackage[utf8]{inputenc}
%\usepackage{kpfonts}
\usepackage{lastpage}
\usepackage{lmodern}
\usepackage{multicol}
\usepackage{multirow} % Allow multirow in tabular env
\usepackage{rotating}
\usepackage{setspace}
%\usepackage{subfigure} %% deprecated ?
\usepackage{subcaption}
\usepackage{verbatim}
\usepackage{dsfont}
\usepackage{gensymb}




\usepackage{amsmath}
\usepackage{amssymb}
\usepackage{pgfplots}

\pgfplotsset{compat=1.9}



\geometry{top=1.5cm, bottom=1.5cm, left=1.5cm, right=1.5cm}

%\newcommand\textsubscript[1]{\ensuremath{{}_{\text{#1}}}}
\newcounter{toto}
\newcommand{\pn}{\par\noindent}
\newcommand{\vs}{\vskip 0.5 \baselineskip}
\newcommand{\la}{\langle}
\newcommand{\ra}{\rangle}
%
\newcommand{\HRule}{\rule{\linewidth}{0.5mm}}
\newcommand{\bm}{\boldmath}
\newcommand{\um}{\unboldmath}
\newcommand{\hm}{\fontfamily{phv}\fontseries{m}\selectfont}
\newcommand{\hb}{\fontfamily{phv}\fontseries{b}\selectfont}
\newcommand{\cf}[1]{{\tt (cf. : #1)}}
%
\def\div{\mathop{\rm div}\nolimits}
\def\grad{\mathop{\overrightarrow{\rm grad}}\nolimits}
\def\rot{\mathop{\overrightarrow{\rm rot}}\nolimits}
\def\u{\mathop{\vec u}\nolimits}
%
%
%
%%%%%%%%%%%%%%%%%%%%%%%%%%%%%%%%%%%%%%%%%%%%%%%%%%%%%%%%%%%%%%%%%%%%%%%%%%%%%
%
%
%\renewcommand{\chaptername}{Lecture}
\addto\captionsfrench{\renewcommand{\chaptername}{Partie}}

\linespread{1.1}
\begin{document}
%
\thispagestyle{empty}
\renewcommand{\headrulewidth}{0.0pt}
%

\setlength{\columnsep}{30pt}
\setlength{\columnseprule}{1pt}
%\begin{multicols}{2}
\noindent
%\begin{minipage}[t]{0.35\textwidth}
\begin{minipage}[t]{0.35\textwidth}
~\\[-2ex]
% \includegraphics[width=1.8cm]{/home/spetit/Administration/Logos/logo-sciences-couleur-texte_new}
\includegraphics[scale=0.7]{unicaen_logo_rvb_noir_V1}

\end{minipage}
%\begin{minipage}[t]{0.5\textwidth}
\begin{minipage}[t]{0.50\textwidth}
\flushleft{
\footnotesize{UFR des sciences} \\
%\tiny{~}\\
\footnotesize{Licence PCI -- seconde année} \\
%\tiny{~}\\
\footnotesize{Initiation aux méthodes numériques}
}
\end{minipage}
%\hfill
\begin{minipage}[t]{0.12\textwidth}
\flushleft
{
\footnotesize{2017-2018}}
\end{minipage}

\vspace{0.5cm}

\centerline{\sc\Large Travaux pratiques de programmation}
{\let\clearpage\relax \section*{Introduction}}
Les travaux pratiques qui suivent visent à vous aider à mieux appréhender 
les méthodes, les raisonnements, et les outils de programmation ;
ceux-ci étant largement indispensables à la pratique des sciences à
haut niveau, dans laquelle il est souvent nécessaire
de créer ses propres outils. 

Pour des raisons d'accessibilités la programmation sera faite en Python ;
ce langage étant pratiqué au lycée. Le langage lui-même ne sera pas enseigné ; 
il n'est pas le sujet de ce cours.
À l'aide d'exemples, et par une pratique réitérée, 
vous serez amenés à en développer une connaissance empirique. 
Un fort niveau d'engagement personnel est donc nécessaire en séance. 

Chaque TP emploie des notions simples de mathématique à aborder avec 
les éléments les plus accessibles de Python. Le tout est conçu pour vous
permettre de vous concentrer sur l'aspect essentiel de ces TP :
l'algorithmique. 

% 1 séances 2 exercices
%\newcommand{cf}[1]{({\sc cf. :}{\tt #1.})}
%\subimport{./tp_polynome/}{exo.tex}

\pagebreak
\section{Premiers pas en Python}
\subimport{./tp_polynome/exemples/}{enonce.tex}
\pagebreak
\subimport{./tp_polynome/exercices/}{enonce.tex}
\pagebreak
%\subimport{./tp_stat/}{exo.tex}
\section{Quelques boucles}
\subimport{./tp_stat/exemples/}{enonce.tex}
\pagebreak
\subimport{./tp_stat/exercices/}{enonce.tex}
\pagebreak
\section{Dérivation}
\subimport{./tp_derivation/exercices/}{enonce.tex}
\pagebreak
\section{Intégration}
\subimport{./tp_integration/exercices/}{enonce.tex}
\pagebreak
\section{Évaluation de fonctions}
\subimport{./tp_dl/exercices/}{enonce.tex}
\pagebreak
%\subimport{./tp_aleatoire/}{exo.tex}
\section{Des aléas de l'informatique}
\subimport{./tp_aleatoire/exemples/}{enonce.tex}
\pagebreak
\subimport{./tp_aleatoire/exercices/}{enonce.tex}
\pagebreak
\subimport{./tp_zero/exercices/}{enonce.tex}





\end{document}


