%\documentclass[a4paper,12pt]{article}
%\documentclass[a4paper,landscape,11pt]{report}
\documentclass[a4paper,11pt,titlepage]{article}
%
\usepackage{amsmath}
\usepackage{amssymb}
\usepackage{array}
\usepackage[french]{babel}
\usepackage{calc}
%\usepackage{caption}
\usepackage[hang]{caption}
\usepackage{cite}
\usepackage{color}
\usepackage{dsfont}
\usepackage{enumerate}
\usepackage{enumitem}
\usepackage{epic}
\usepackage{eepic}
\usepackage{eepicemu}
\usepackage{epsfig}
\usepackage{fancyhdr}
\usepackage{float}
\usepackage[T1]{fontenc}
% \usepackage{fourier}
\usepackage{gensymb}
\usepackage[dvips]{geometry}
\usepackage{graphicx}
\usepackage{hyperref}
\usepackage{import} % To use \subimport and import stuff
\usepackage[utf8]{inputenc}
%\usepackage{kpfonts}
\usepackage{lastpage}
\usepackage{listings} % to include code in latex
\lstset
{ %Defaults formatting for code included in lstlisting
    language=Python,
    basicstyle=\footnotesize,
    numbers=left,
    stepnumber=1,
    showstringspaces=true,
    tabsize=3,
    breaklines=true,
    breakatwhitespace=false,
    inputencoding=utf8,
    extendedchars=true,
    literate={à}{{\`a}}1 {À}{{\`A}}1 {é}{{\'e}}1  {è}{{\`e}}1 {É}{{\'E}}1,
    frame=L,
    xleftmargin=\parindent,
}

\usepackage{lmodern}
\usepackage{multicol}
\usepackage{multirow} % Allow multirow in tabular env
\usepackage{placeins}
\usepackage{pgfplots}
\usepackage{rotating}
\usepackage{setspace}
%\usepackage{subfigure} %% deprecated ?
\usepackage{subcaption}
\usepackage{verbatim}





%\geometry{top=2.0cm, bottom=2.0cm, left=2.0cm, right=2.0cm}
%\geometry{top=3cm, bottom=3cm, left=2.5 cm, right=2.5cm}



\pgfplotsset{compat=1.9}





%\newcommand\textsubscript[1]{\ensuremath{{}_{\text{#1}}}}
\newcounter{toto}
\newcommand{\pn}{\par\noindent}
\newcommand{\vs}{\vskip 0.5 \baselineskip}
\newcommand{\la}{\langle}
\newcommand{\ra}{\rangle}
%
\newcommand{\HRule}{\rule{\linewidth}{0.5mm}}
\newcommand{\bm}{\boldmath}
\newcommand{\um}{\unboldmath}
\newcommand{\hm}{\fontfamily{phv}\fontseries{m}\selectfont}
\newcommand{\hb}{\fontfamily{phv}\fontseries{b}\selectfont}
\newcommand{\cf}[1]{{\tt (cf. : #1)}}
%
\def\div{\mathop{\rm div}\nolimits}
\def\grad{\mathop{\overrightarrow{\rm grad}}\nolimits}
\def\rot{\mathop{\overrightarrow{\rm rot}}\nolimits}
\def\u{\mathop{\vec u}\nolimits}
%
%
%
%%%%%%%%%%%%%%%%%%%%%%%%%%%%%%%%%%%%%%%%%%%%%%%%%%%%%%%%%%%%%%%%%%%%%%%%%%%%%
%
%
%\renewcommand{\chaptername}{Lecture}
\addto\captionsfrench{\renewcommand{\chaptername}{Partie}}

\linespread{1.1}
\begin{document}



\begin{titlepage}
  ~\\[-12ex]
  \begin{tabular}{lll} 
    \raisebox{-.6\height}{
    \begin{minipage}[t]{0.35\textwidth}
      \includegraphics[scale=0.8]{unicaen_logo_rvb_noir_V1}
    \end{minipage}}&
                     \begin{minipage}[t]{0.40\textwidth}
                       \flushleft{
                         \footnotesize{UFR des sciences\strut} \\
                         % \tiny{~}\\
                         \footnotesize{Licence PCI -- seconde année\strut} \\
                         % \tiny{~}\\
                         \footnotesize{Calcul scientifique 1\strut}
                       }
                     \end{minipage}&
                     % \hfill
                                     \begin{minipage}[t]{0.12\textwidth}
                                       \flushleft
                                       {
                                         \footnotesize{2019--2020}}
                                     \end{minipage}
  \end{tabular}
  \par
  \centering
  \vspace{12\baselineskip}
  {\Huge 
    Travaux pratiques de programmation\strut\\ en Python \strut\\ \strut Initiation au calcul scientifique\par}
  \vspace{4\baselineskip}
  \par
  {\Large\textsc{Yves Lemière, François Mauger,\\ et Pierre-Matthieu Anglade}\par}
  %{\texttt{(lemiere@lpccaen.in2p3.fr,\\mauger@lpccaen.in2p3.fr,\\pierre-matthieu.anglade@unicaen.fr)}\par}
   \vfill
  % in order to get a fancy degree at\par
   {\em lemiere@lpccaen.in2p3.fr,\\mauger@lpccaen.in2p3.fr,\\pierre-matthieu.anglade@unicaen.fr}
  % \end{addmargin}
\end{titlepage}
\leavevmode\thispagestyle{empty}\newpage



\begin{abstract}
  Les travaux  pratiques d'initiation au calcul  scientifique ont pour
  objectif  de  vous  aider  à mieux  appréhender  les  méthodes,  les
  raisonnements,  et les  outils  de programmation  pour réaliser  des
  calculs  sur  ordinateur.   De  telles  compétences  sont  largement
  indispensables  à  la pratique  des  sciences  à haut  niveau pour
  lesquelles il est souvent nécessaire de créer ses propres outils pour
  déterminer  numériquement  les solutions de problèmes calculatoires,
  analyser et manipuler des données scientifiques.

  Pour des  raisons d'accessibilité, on s'exercera  à la programmation
  avec le  langage Python, déjà utilisé  au lycée par exemple  dans le
  cadre de l'enseignement ISN du Bac série S (Informatique et Sciences
  du  Numérique).   On  ne  présentera que  les  éléments  simples  et
  immédiatement  utilisables du  langage  Python, l'acquisition  d'une
  maîtrise experte de ce langage n'étant pas le sujet de ce cours.

  À  l'aide  d'exemples, et  par  une  pratique réitérée,  vous  serez
  amenés,  encadrés   par  votre  enseignant,  à   en  développer  une
  connaissance  empirique, pratique  mais  éclairée.   Un fort  niveau
  d'engagement  personnel est  indispensable en  séance, mais aussi
  entre celles-ci. 

  Chaque TP  emploie des  notions simples  de mathématiques  à aborder
  avec les éléments les plus accessibles  de Python. Le tout est conçu
  pour vous permettre de vous concentrer sur l'aspect essentiel de ces
  TP : Le raisonnement sur les états de la machine et l'algorithmique.
  
\end{abstract}





% 1 séances 2 exercices
%\newcommand{cf}[1]{({\sc cf. :}{\tt #1.})}
%\subimport{./tp_polynome/}{exo.tex}

\pagebreak
\section{Premiers pas en Python}
\subimport{./tp_premiersPas/exercices/}{enonce.tex}
\pagebreak

\section{Racines d'un polynôme du second degré}
\subimport{./tp_polynome/exercices/}{enonce.tex}
\pagebreak

\section{Boucles et graphiques}
\subimport{./tp_bouclesGraphes/exercices/}{enonce.tex}
\pagebreak




%\subimport{./tp_stat/}{exo.tex}
%\section{Pour aller plus loin}
%\subimport{./tp_stat/exemples/}{enonce.tex}
%\pagebreak


\section{Dérivation}
\subimport{./tp_derivation/exercices/}{enonce.tex}
\FloatBarrier
\pagebreak

\section{Intégration}
\subimport{./tp_integration/exercices/}{enonce.tex}
\pagebreak

\section{Évaluation de fonctions}
\subimport{./tp_dl/exercices/}{enonce.tex}
\pagebreak

% Recherche des zéros d'une fonction
\subimport{./tp_zero/exercices/}{enonce.tex}
\pagebreak
\section{Écriture et lecture de données [Facultatif]}
\subimport{./tp_io/exercices/}{enonce.tex}
\pagebreak
\section{Statistiques[Facultatif]} %Nécessite d'avoir fait tp_io
\subimport{./tp_stat/exercices/}{enonce.tex}
\pagebreak


%\subimport{./tp_aleatoire/}{exo.tex}
\section{Des aléas dans la machine [Facultatif]}
\subimport{./tp_aleatoire/exemples/}{enonce.tex}
\pagebreak

\subimport{./tp_aleatoire/exercices/}{enonce.tex}
\pagebreak







\end{document}
