%%\section{Exercices}

Cette section propose de mettre en pratique certaines
connaissances acquise à la section précédente.

À partir d'un état donné, les trois coefficients d'un polynôme
du second degrés, il faudra écrire un programme manipulant
cet état pour faire afficher la ou les racines de ce
polynôme.

Pour rappel, soit  $f(x) = a.x^2+b.x+c$, où $a$, $b$, et $c$ sont les 
 coefficients   de  l'équation  avec  $a$
 non nul. La résolution de l'équation $f(x) = 0$
 nécessite le calcul du
 discriminant $\Delta =  b^2-4ac$.

 Si  le 
discriminant est :
\begin{description}
  %%
\item[positif], alors il existe deux solutions réelles :
\begin{eqnarray*}
  x_1   =  \frac{-b-\sqrt{\Delta}}{2.a}   \qquad  \textrm{et}   \qquad
  x_2=\frac{-b+\sqrt{\Delta}}{2.a}\,;
\end{eqnarray*}
%%
\item[nul], alors il existe une solution unique réelle :
\begin{eqnarray*}
  x_1 = \frac{-b}{2.a}\,;
\end{eqnarray*}
%%
\item[négatif], alors il existe deux solutions complexes :
\begin{eqnarray*}
  x_1  =   \frac{-b-i\sqrt{\Delta}}{2.a}  \qquad   \textrm{et}  \qquad
  x_2=\frac{-b+i\sqrt{\Delta}}{2.a}.
\end{eqnarray*}
%%
\end{description}

\subsection{Programmation de la résolution}



Écrire votre code en suivant ces étapes pour déterminer le
nombre et les valeur des solutions de l'équation :
\begin{enumerate}
%%

\item Faire intervenir un utilisateur auquel sera demandé d'entrer
  les valeurs des paramètres \texttt{a}, \texttt{b} et \texttt{c}.
%%
\item Afficher chacune des variables pour vérification.
%%
\item  Une fois  les  valeurs affectées à chaque variable, calculer  le
  discriminant.\\ Penser à importer les modules nécessaires !
%%
\item  Tester le  résultat du  discriminant  puis, en  fonction de  cette
  valeur, aficher un  message à l'écran informant du nombre
  de solutions existantes.
%%
\item Calculer  les racines de l'équation et afficher le résultat.
  %%
\item Vérifier \emph{manuellement} les résultats fournis par le programme.

\end{enumerate}
