\subsection{Écrire dans un fichier}


Pour s'exercer à écrire des messages textuels dans un fichier, vous étudierez le fichier \texttt{write\_file.py}.

\begin{enumerate}

\item La ligne 21 permet d'ouvrir le fichier \texttt{quote\_file.dat} en écriture grâce
  à l'option \texttt{w} (\emph{write}).
  Complétez le programme des lignes nécessaires pour ajouter, à la fin du
  fichier \texttt{quote\_file.dat}, la traduction française de la citation.

\item   Écrivez   un  nouveau   programme   \texttt{write\_file\_2.py}
  permettant d'ouvrir  de nouveau le  fichier \texttt{quote\_file.dat}
  afin d'y ajouter une nouvelle citation. Pour cela, il faut savoir que l'option \texttt{a} (\emph{append}) permet
  de préciser qu'un fichier est ouvert en écriture ET que les nouvelles données seront écrites après celles déjà présentes. 

\end{enumerate}

\begin{figure}  
  \lstinputlisting{../exemples/write_file.py}
  \caption{Contenu du fichier \texttt{write\_file.py}}
  \label{stat_write_file}
\end{figure}

\subsection{Lire un fichier texte}


Pour  s'exercer à  lire des données à partir d'un  fichier dans un format textuel,
vous  étudierez le  programme \texttt{read\_file.py}.

\begin{enumerate}

\item Expliquez  les différences  entre les  méthodes \texttt{read()},
  \texttt{readlines()} et \texttt{readline()}.

\item  Comment peut-on lire  l'ensemble  du fichier  avec la  méthode
  \texttt{readline()} ?

\item  Que se  passe-t-il lorsque  l'on \textit{oublie}  de fermer  le
  fichier à la fin du programme ?

\item Testez le type auquel appartient la variable \texttt{lines}.
% Quels genres de tests t'attends-tu à ce que les étudiants fassent ?

\item Lors de l'affichage de  la variable \texttt{lines}, chaque ligne
  est encadrée par les caractères \texttt{'} (\emph{quote}) et les différentes
  lignes se trouvent
  séparées par des  virgules. L'ensemble est délimité  par des crochets
  \texttt{[]}.  Chaque  élément séparé d'un autre par une  virgule est accessible
  indépendamment.  Cet objet est de  type \texttt{list}, c'est un type
  de  variable \emph{container}  qui  peut  contenir   plusieurs  variables accessibles
  au moyen d'un indice. Cette variable se comporte comme un tableau unidimensionnel
  composé de cases juxtaposées dans lesquelles on peut stocker d'autres variables.
  Commentez  les lignes  49 et  50 ; puis proposez  deux méthodes  pour
  accéder à l'ensemble des éléments de l'objet \texttt{lines}

\end{enumerate}


\begin{figure}  
  \lstinputlisting{../exemples/read_file.py}
  \caption{Contenu du fichier \texttt{read\_file.py}}
  \label{stat_read_file}
\end{figure}
