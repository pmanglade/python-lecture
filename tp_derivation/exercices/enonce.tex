\section{Derivation}
Il arrive que la dérivation analytique d'une fonction ne soit pas facilement
calculabe. Ce TP survol les méthodes de dérivation numérique.
\subsection{Précision}
Cet exercice vise à étudier les erreurs commises lors de l'emploi
d'un méthode de dérivation numérique.
\begin{enumerate}
\item Écrire un fonction capable de calculer la dérivée 
d'une fonction (de $\R & \longrightarrow & \R$)  en 
un point $x$ quelconque en utilisant la formule : 
$$f'(x)= \frac{f(x+h)-f(x-h)}{2h}\,;$$
où $h$ est un paramètre qui pourra être choisi lors de l'appel de cette fonction.
\item Utiliser cette fonction pour calculer $\frac{d\cos(x)}{dx}$ 
sur mille points répartis homogènements sur $[0\,;\,\,2\pi]$.
\item Sur les mille points, calculer la moyenne de la valeur
 absolue de la différence.
\item Sur une échelle log/log, tracer l'évolution de cette différence 
en fonction de $h$. Les valeurs choisies pour $h$ suivront une suite 
géométrique de raison $2^\frac{1}{2}$ et de valeur initiale $10^{-6}$. 
On utilisera les 25 premières valeurs.
Qu'observe-t-on ? 
Comment choisir la meilleur valeur de $h$ pour dériver numériquement ? 
\item Tracer la dérivée numérique et la dérivée 
analytique sur un même graphique. Peut-on observer facilement la différence ? 
\item Tracer la différence ente dérivée numérique et calcul exact. 
\item Comparer la forme de la fonction obtenue avec 
$-\alpha \sin(x)$ en choisissant de manière appropriée $\alpha$.
Expliquer ce qui est observé. 
\end{enumerate}

\subsection{Algorithmes de dérivation}
De nombreux algorithmes de dérivation numérique existent. Cette section vise à
en programmer quelques-uns et à les comparer.
\begin{enumerate}
\item Programmer les fonctions de dérivation utilisant les formules : 
$$f'(x)= \frac{-f(x+2h)+4f(x+h)-4f(x-h)+f(x-2h)}{12h}\,;$$ 
$$f'(x)= \frac{f(x+3h)-9f(x+2h)+45f(x+h)-45f(x-h)+9f(x-2h)-f(x-3h)}{60h}.$$
Les coefficients de ces fonctions sont calculés de manière à annuler de plus en
plus de termes du développement limité de la fonction $f$.
\item Des formules non-symétriques existent également. En voici quelques-unes :
 $$f'(x) = \frac{f(x+h)-f(x)}{h}\,;$$

 Programmer les fonctions de dérivation correspondantes.


\end{enumerate}


\subsection{Ordres supérieurs}

\subsection{Dérivées de séries de points}
