%%
Il  arrive que  la dérivation  analytique d'une  fonction ne  soit pas
facilement  calculable.  Ce TP  survole  des  méthodes élémentaires  de
dérivation numérique.

\subsection{Précision}
Cet exercice vise à étudier les erreurs commises lors de l'emploi d'une
méthode de dérivation numérique. La méthode décrite ici est
la plus simple des méthodes dites par \emph{différences finies}. 
\begin{enumerate}
  %%
\item Écrire une fonction capable de calculer la dérivée d'une fonction
  (de  $\mathbb{R}  \longrightarrow  \mathbb{R}$)   en  un  point  $x$
  quelconque en utilisant la formule :
  \begin{equation}
    f'(x) \simeq D^f(x) = \frac{f(x+h)-f(x-h)}{2h}\,; \label{eqDiffFinies}
  \end{equation}
  où $h$ est un  paramètre qui
  pourra   être   choisi   lors   de   l'appel   de   cette   fonction
  \cf{function.py}.  %%
\item Utiliser cette fonction  pour calculer
  $D^{\cos} (x)\approx \frac{d\cos(x)}{dx}$ sur
  $n=100$ points $x_i$ répartis uniformément sur $[0\,;\,\,2\pi]$ \cf{mesh.py}.
  Ensuite tracer la  fonction $\cos(x)$ et les points de $D^{\cos} (x)$
  calculés sur un
  graphique \cf{plot.py}.  %%
\item  Pour ce même lot de 100 points,  calculer   les  valeurs   de
  $\frac{d\cos(x)}{dx}   =   -\sin(x)$   puis  faire   la   différence
  entre dérivée numérique et dérivée analytique :
  $$\varepsilon(x) = D^{\cos}(x)+\sin(x)$$%%
\cf{loop.py}.
\item Calculer la moyenne
  de la valeur  absolue de  cette différence
  $\bar{\varepsilon} = \frac{1}{n}\sum_n |\varepsilon(x_i)|$.
\item Avec  une échelle  log/log \cf{plot.py},
  tracer  l'évolution de  cette différence
  $\bar{\varepsilon}$         en        fonction         de la
  valeur choisie pour $h$ lors de la dérivation par différences finies. 
  On aura donc une fonction calculant $\bar{\varepsilon}(h)$.

  Les calculs décrits précédemment 
  devront donc être fait pour un (grand) ensemble de valeur de $h$ : ${h_i}$ :
  penser à programmer des fonctions afin de réutiliser les instructions.
  
  Les valeurs choisies pour $h$ suivront
  une  suite  géométrique  de  raison  $2^\frac{1}{2}$  et  de  valeur
  initiale  $10^{-6}$ \cf{loop.py}.   On  utilisera  les 25  premières
  valeurs.  Qu'observe-t-on ?  Comment  choisir la meilleure valeur de
  $h$ pour dériver numériquement ?
\end{enumerate}

Les questions précédentes nous ont permis de mettre en évidence
les erreurs liées à l'utilisation de nombre à vrigule \emph{flottante}
pour faire des calculs numériques. Si la précision des calculs en machine
avait été infinie, on aurait observé que :
$\lim_{h\rightarrow 0}\bar{\varepsilon}(h) = 0$.

Dans les questions qui suivent nous allons reprendre l'analyse de la fonction
$D_{\cos} (x)$
afin de montrer que les erreurs commises sont parfaitement analysables
à l'aide de développement limités.

\begin{enumerate}[resume]
\item  Tracer la  dérivée numérique $D^{\cos} (x)$
  (pour une  valeur de  $h$ choisie
  arbitrairement, par exemple $h=0,001$)
  et    la   dérivée   analytique  de $\cos(x)$  sur    un   même
  graphique. Peut-on observer facilement la différence ?
\item  Tracer   la  différence   entre  dérivée  numérique   et  calcul
  analytique.
\item Comparer la forme de  la fonction obtenue avec $-\alpha \sin(x)$
  en choisissant de manière appropriée $\alpha$.
\item Écrire la dérivée à l'ordre 3 de $\cos(x)$.
  Ensuite, expliquer ce qui est
  observé en utilisant les développement limités à l'ordre 5 autour de $x$
  de $f(x\pm h)$ et en insérant ces expressions dans la formule de différences
  finies (éq.~\ref{eqDiffFinies}). 
\end{enumerate}

\subsection{Autres formules de dérivation}

De  nombreuses  formulations  de  dérivation  numérique  existent.  Cette
section vise à en programmer quelques-unes et à les comparer.
\begin{enumerate}
\item Programmer les fonctions de dérivation utilisant les formules suivantes :
  \begin{eqnarray*}
 D^f_2(x) &=& \frac{-f(x+2h)+8f(x+h)-8f(x-h)+f(x-2h)}{12h}\,; \\
  D^f_3(x)&=&\frac{f(x+3h)-9f(x+2h)+45f(x+h)-45f(x-h)+9f(x-2h)-f(x-3h)}{60h}.
  \end{eqnarray*}
Les coefficients de ces fonctions sont calculés de manière à annuler de plus en
plus de termes du développement limité de la fonction $f$ (l'obtention des ces formules
sera étudié en cours de
calcul scientifique 2).
\item Des formules non-symétriques autour de $x$ existent également. En voici quelques-unes :
   \begin{eqnarray*}
D_d^f(x) &=& \frac{f(x+h)-f(x)}{h}\,;\\
D_{d1}^f(x) &=& \frac{-3f(x)+4f(x+h) -f(x+2h)}{2h}\,;\\
D_{d2}^f(x) &=& \frac{-15f(x)+16f(x+h) -f(x+4h)}{12h}.
  \end{eqnarray*}
 Programmer les fonctions de dérivation correspondantes. Ces fonctions peuvent être utilisées avec $h>0$ ou $h<0$.
\item  {\sc [facultatif]} À nombre  d'évaluations de  la fonction  $f$ égaux,  comparer la
  précision moyenne de chacune de  ces méthodes sur un graphique (pour
  des valeurs de $h$ comprises entre $0,001$ et $1$). Pour cela on suivra à chaque fois le plan de l'exercice du
  début de cette section :
  \begin{enumerate}
  \item choisir une fonction à dériver ;
  \item choisir un intervalle et son échantillonnage ;    
  \item pour chaque fonction de dérivation définir et calculr  
    $\bar{\varepsilon}(h)/M$, où $M$ est le nombre d'évaluations de
    la fonction $f$ utilisées pour le calcul de la dérivée de $f$ en
    un point.
  \item Tracer l'évolution des fonctions précédentes en fonction de $h$. 
    \end{enumerate}
\end{enumerate}



\subsection{Ordres supérieurs \sc{[Facultatif]}}
Il est possible  d'obtenir des dérivées d'ordre supérieur  par la même
méthode (seconde, troisième\ldots).
\begin{enumerate}
\item Programmer une  fonction calculant la dérivée à  l'ordre 2 d'une
  fonction        en       utilisant        l'algorithme       $f'(x)=
  \frac{f(x+h)-f(x-h)}{2h}\,;$  pour  obtenir des  dérivées  premières,
  puis  calculant la  dérivée secondes  à partir  de deux  évaluations ainsi
  obtenues.
\item Programmer une fonction calculant la dérivée seconde à l'aide de
  la formule :
$$f'(x)=  \frac{f(x+h)+f(x-h)-2f(x)}{h^2}.$$  Comparer l'évolution  de
  l'erreur de ces deux méthodes en  fonction de $h$. Puis réaliser une
  comparaison \og{} sur papier \fg{} des deux méthodes.
\end{enumerate}



% \subsection{Dérivées d'une série de points régulière \sc{[Facultatif]}}

% Ces méthodes de dérivation numérique  peuvent être utiles pour dériver
% des fonctions dont la forme analytique est complexe et pour lesquelles
% on dispose juste d'une série de points (un échantillon).
% À titre d'exercice nous allons
% générer une  telle série de points  dans un fichier, puis  calculer la
% dérivée.

% \begin{enumerate}
% \item Écrire un programme générant  un fichier contenant deux colonnes
%   \cf{file.py}. Dans  la première  on trouvera les  valeurs de  $x$ et
%   dans la seconde celles de $y=f(x)$ ; où $f$ sera une fonction de votre
%   choix. Générer  cent points ($x$,$y$) sur un  intervalle $[a\,;\,b]$  de votre
%   choix.
% \item Écrire un autre programme capable de relire ce fichier.
% \item  Combien de  valeurs de  la  dérivée peut-on  calculer avec  une
%   formule de dérivation centrée ?
% \item  Écrire  une fonction  capable  de  calculer  la dérivée  de  la
%   fonction échantillonnée dans le fichier pour chacun des points.
% \end{enumerate}


%\begin{figure}
%\lstinputlisting{../exemples/file.py}\caption{file.py}
%\end{figure}

\begin{figure}
\lstinputlisting{../exemples/function.py}\caption{function.py}
\end{figure}

\begin{figure}
\lstinputlisting{../../tp_integration/exemples/if.py}\caption{if.py}
\end{figure}

\begin{figure}
\lstinputlisting{../../tp_integration/exemples/listElement.py}\caption{listElement.py}
\end{figure}

\begin{figure}
\lstinputlisting{../../tp_integration/exemples/listOfList.py}\caption{listOfList.py}
\end{figure}

\begin{figure}
\lstinputlisting{../exemples/loop.py}\caption{loop.py}
\end{figure}

\begin{figure}
\lstinputlisting{../exemples/mesh.py}\caption{mesh.py}
\end{figure}

\begin{figure}
\lstinputlisting{../exemples/plot.py}\caption{plot.py}
\end{figure}

\begin{figure}
\lstinputlisting{../../tp_dl/exemples/polynomesEtSeries.py}\caption{polynomesEtSeries.py}
\end{figure}

\begin{figure}
\lstinputlisting{../../tp_integration/exemples/returnList.py}\caption{returnList.py}
\end{figure}
\pagebreak
