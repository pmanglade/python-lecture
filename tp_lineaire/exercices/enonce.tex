%%

D'après Wikipedia \og{}L’algèbre linéaire est la branche des mathématiques
qui s'intéresse aux espaces vectoriels et aux transformations linéaires,
formalisation générale des théories des systèmes d'équations linéaires.\fg{}
C'est aussi l'un des domaines clefs du calcul scientifique car nombre de
problèmes peuvent être simulés/résolus grâce à la représentation de
grandeur physique dans des espaces multidimensionnels
(souvent des milliars de dimensions).

Ces aspects de simulation de physique en calcul scientifique sont
toutefois hors de la portée de ce cours. Ici nous nous contenterons
d'utiliser une méthode de résolution élémentaires appliquée à
des systèmes d'équations. L'objectif étant essentiellement d'apprendre
à bien manipuler les listes et boucles vues précédemment.

\subsection{Petit système d'équations}
\begin{enumerate}
  \item Soit $b=(11,5,13)$, \[a=\left(
        \begin{array}{ccc}
          3 & 2 & -1 \\
          1 & 2 & -1 \\
          2 & 4 & 5 
          \end{array}
        \right), \]
      et $x = (x_0, x_1, x_2)$, écrire le système d'équation correspondant à
      $$ b=ax.$$
  
\item En procédant à des combinaisons bien choisies de lignes résoudre
  manuellement le système d'équations suivant :
  \[
    \left\{
      \begin{array}{rcl}
        11 & = & 3x_0 + 2x_1 -x_2 \\
        5  & = & x_0 + 2x_1 -x_2 \\
        17 & = & 2x_0+4x_1 + 5x_2
      \end{array}
    \right.
  \]
  De bons choix permettent de supprimer plusieurs inconnues d'un seul coup.
\item Représentation en machine du système d'équation :
  \begin{enumerate}
  \item Écrire un programme \verb+lineaire1.py+
  \item Y déclarer une liste \verb+b=[11,5,17]+.
  \item Y déclarer une liste de liste \verb+a=[[3,2,-1],[1,2,-1],[2,4,5]]+.
  \item À quoi correspondent les chiffres de la liste imprimées en faisant
    \verb+print(a[0])+, en faisant \verb+print(a[1])+ ?
  \item Trouver comment faire imprimer le second élément de \verb+a[2]+.
  \item À l'aide d'une boucle \verb+for+ remplacer les éléments \verb+a[0][i]+
    par ceux obtenues en calculant \verb+a[0][i] - a[1][i]+. De même remplacer
    \verb+b[0]+ par \verb+b[0]-b[1]+.
  \item Faire imprimer \verb+a+ et \verb+b+ par le programme. Expliciter en terme
    de combinaison de lignes du sytème précédent l'opération réalisée. 
  \end{enumerate}
\item Résolution en machine du sytème :
   \begin{enumerate}
   \item Continuer le programme. Cette fois programmer le calcul de la différence
     entre la ligne 3 et 2 fois la ligne 2 du système. Le résultat viendra prendre la
     place de la ligne 3.
   \item Soit $a'$ la matrice et $b'$ le vecteur obtenus suite à ces deux combinaisons de lignes.
     Constater que $x_0$ et $x_2$ peuvent être calculés aisément puisque $b'=a'x$.
   \item Les faire calculer par le programme.
   \item En utilisant les valeurs de \verb+a[1]+, \verb+b[1]+, et les deux éléments de solution
     déjà trouvés, programmer le calcul de $x_1$.
   \item Insérer les trois valeurs trouvées dans une liste \verb+x+ et les faire imprimer par le programme.
     
  \item Que se passe-t-il si on ré-exécute le même programme en ajoutant au début
     (après avoir remplis \verb+b+) \verb+b[2] = 24+ ? Les valeurs trouvées pour $x_1$, $x_2$, et $x_3$
     sont-elles solutions du système d'équations suivant :
     \[
       \left\{
         \begin{array}{rcl}
           11 & = & 3x_0 + 2x_1 -x_2 \\
           5  & = & x_0 + 2x_1 -x_2 \\
           24 & = & 2x_0+4x_1 + 5x_2
         \end{array}
       \right. \,?
     \]    
  
   \item Que se passe-t-il si on ré-exécute le même programme en ajoutant au début
     (après avoir remplis \verb+a+) \verb+a[1][2] = 1+ ? Les valeurs trouvées pour $x_1$, $x_2$, et $x_3$
     sont-elles solutions du système d'équations suivant :
     \[
       \left\{
         \begin{array}{rcl}
           11 & = & 3x_0 + 2x_1 -x_2 \\
           5  & = & x_0 + 2x_1 +x_2 \\
           17 & = & 2x_0+4x_1 + 5x_2
         \end{array}
       \right. \,?
     \]    
     
   \end{enumerate}
 \end{enumerate}

 \subsection{Pivot de Gauss simplifié}
 Dans la section précédente nous avons programmé une résolution
 analytique d'un système d'équation et surtout souligné ---
 ce que l'on n'ignorait pas --- le fait
 qu'une telle résolution est dépendante des coefficients de $a$.
 Le programme \verb+lineaire1.py+ est donc de peu d'utilité puisque
 adapté uniquement à calculer les solutions d'un système déjà résolu.

 Pour obtenir un programme utile, il nous faut une méthode fonctionnant
 indépendamment (ou presque) des coefficients de $a$. Cet algorithme
 est une version simplifiée du pivot de Gauss.

 Notons $a_{ij}$ les éléments de $a$. Si pour toute les lignes $i$ ($i>0$)
 on remplace la ligne par elle-même moins $a_{i0}/a_{00}$ fois la ligne zéro,
 on obtient une matrice $a'$ de la forme :
 \[
   a' = \left(
     \begin{array}{ccc}
       a_{00} & a_{01} & a_{02} \\
       0 & a'_{11} & a'_{12} \\
       0 & a'_{21} & a'_{22}       
     \end{array}
   \right).
 \]

 On peut réitérer ce processus en retirant à toutes les lignes $i$ ($i>1$)
 $a_{i1}/a_{01}$ fois la ligne 1.

 La matrice $a''$ obtenue aura alors la forme suivante :
  \[
   a'' = \left(
     \begin{array}{ccc}
       a_{00} & a_{01} & a_{02} \\
       0 & a'_{11} & a'_{12} \\
       0 & 0 & a''_{22}        
     \end{array}
   \right).
 \]

 Dans le cas de notre matrice 3x3 nous disposons alors d'une forme triangulaire et il est facile d'obtenir
 les solutions du système d'équation $b''=a''x$.

 Tel que décris ici l'algorithme est celui d'un pivot de Gauss simplifié essentiellement
 car il est limité à un système de trois
 équations à trois inconnus et parce qu'il ne gère pas le cas où un coefficient nul serait sur la diagonal de la matrice. 

 \begin{enumerate}
 \item Programmer chacune des étapes de l'algorithme précédent dans un fichier nommé \verb+lineaire2.py+.
 \item Vérifier que le programme obtenu trouve effectivement les solutions du système
   d'équation pour différentes valeurs de $a$ et $b$.
 \item {\sc[Facultatif :]} Modifier le programme de sorte qu'il puisse
   fonctionner pour des systèmes avec $n$ équations et inconnues.
 \item {\sc[Facultatif :]} Pour rendre plus efficace et polyvalent notre programme il suffit de lui ajouter
   une étape avant chaque modification des lignes :

   Avant la ième modification
   des lignes, trouver parmi les lignes de numéro $l\geq i$ celle dont $|a_{li}|$ est le plus grand ; échanger
   les lignes $l$ et $i$.

   Cela permet d'éviter les divisions par zéro
   lorsqu'il existe une unique solution au système. 
 \end{enumerate}



\pagebreak
