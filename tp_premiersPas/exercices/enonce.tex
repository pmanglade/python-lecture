%%\section{Exemples}

\subsection{Découverte de l'interpréteur python}

Pour ces quelques questions introductives, utiliser l'interpréteur de
commande Python.

\begin{enumerate}

\item Dans un terminal, taper la commande \texttt{python3}. Lire ce qui s'affiche.

\item Utiliser l'invite de commande  de cette console Python comme une
  calculatrice. Entrer par exemples \verb-2+2- ou \verb+4*3+.

\item Affichez votre premier texte  à l'écran en utilisant la fonction
  print : \\ \texttt{print("Hello World")}

\item Quittez l'interpréteur Python en effectuant \texttt{Ctrl+D}

\end{enumerate}

\noindent \textbf{Remarque:} C'est la version 3 de Python que nous utiliserons.
Pour utiliser la version 2 (non recommandée pour cet enseignement) du langage et de son interpréteur,
on tapera la commande \texttt{python2}.


\subsection{Un peu plus qu'une calculatrice}

Pour   cette  série   de   commandes, commencer  par   ouvrir
l'interpréteur de commandes Python.

\begin{enumerate}


\item        Déclarer        plusieurs        variables        nommées
  \texttt{a}, \texttt{b}, \texttt{c}, \texttt{d},      \texttt{e}      et
  \texttt{f} en les initialisant avec les valeurs suivantes : \texttt{5}, \texttt{13},
  \texttt{17}, \texttt{3.14159} (le séparateur décimal est le point),
  \texttt{'bob'} et \texttt{"bonjour"}.\\ Exemple : \texttt{a = 5}



\item  Affichez  à l'écran  la  valeur  de  chacune des  variables  en
  utilisant la commande \texttt{print}. Remarquer que chaque variable
  persiste dans la mémoire de l'ordinateur. D'une ligne sur l'autre
  elle peuvent être réutilisées. 

\item  Afficher   le  type  de   chaque  variable  avec   la  commande
  \texttt{type()}.\\ Exemple : \texttt{type(a)}

\item Calculer la somme des variables \texttt{a} et \texttt{b} et affecter
  le  résultat  à  une   variable  \texttt{sum}. Faire afficher celle-ci.

\item Affecter la valeur \texttt{666} à la variable \texttt{a} puis afficher de
  nouveau la valeur de la variable \texttt{sum}.

\item  Faire  le calcul  de  \texttt{a}  divisé par  \texttt{b}.

\item  Faire  le calcul  de  \texttt{a}  divisé par  \texttt{d}.

\item  Faire  la  différence  de \texttt{f}  et  de  \texttt{c}.

\item Que donne la somme des variables \texttt{f} et \texttt{e} ?

%\item  Proposez une  méthode pour  échanger le  contenu des  variables
%  \texttt{a}   et   \texttt{b}   (n'utilisez   pas   de   langage   de
%  programmation)!  Réalisez  ensuite  cette méthode  en utilisant  les
%  commandes Python que vous connaissez.
\item 

\end{enumerate}

Remarques : pour faire des opérations nous avons utilisés
ce qui s'appelle en programmation des opérateurs
(\verb_+_, \verb+*+, \verb+/+\ldots).
L'opérateur le plus important d'un langage est l'opérateur d'affectation \verb+=+.
Il permet de mettre de l'information dans une case de mémoire de l'ordinateur (une variable).
Attention à ne pas le confondre avec le symbole $=$ utilisé en mathématiques qui traduit
la similitude, l'égalité, ou encore l'équivalence de deux expressions. 



\subsection{Premier exercice}
Chaque variable dans un programme Python est une sorte de case
dans la mémoire de la machine. En utilisant le nom de la variable
on peut accéder à la case, y mettre de l'information (par exemple un
nombre), ou utiliser l'information qui y est contenue.
\begin{enumerate}
\item Déclarer deux variables \verb+a+ et \verb+b+ et
  leur affecter, par exemple les valeurs respectives 1, et 2.
\item Considérer ces deux cases de mémoire \verb+a+ et \verb+b+.
  À l'aide de ce qui précède trouver
  une manière d'échanger leurs contenus.
\item La méthode utilisée aurait-elle fonctionnée à l'identique
  si les contenus initiaux avaient été
  différents ? Le tester en affectant
  d'autres valeurs aux variables \verb+a+ et \verb+b+ puis en
  appliquant à nouveau exactement les mêmes instructions que
  précédemment, et en vérifiant ensuite l'inversion ou non.
\item Si votre méthode ne fonctionnait pas pour d'autres
  contenus initiaux, recommencer l'exercice.
  
\item Refaire le même exercice mais en partant cette fois de trois
  variables \verb+a+, \verb+b+, et \verb+c+, dont les contenus
  seront échangés de sorte que \verb+b+ prenne la valeur dans \verb+a+,
  \verb+c+ celle dans \verb+b+\ldots
\end{enumerate}

\subsection{Premier programme Python}

Afin  d'exécuter   une  suite   potentiellement  longue   et  complexe
d'instructions, on peut
préalablement enregistrer ces instructions dans un fichier et demander
ensuite à l'interpréteur Python de  \og{}digérer\fg{} le contenu de ce
fichier.   On  parle   d'exécution  d'un   code  \emph{source},   d'un
\emph{script} ou d'une \emph{macro} Python.

On utilise généralement un programme spécialisé dans l'édition
de code informatique pour créer et éditer un tel fichier
(Environnement de Développement Intégré). Dans notre cas,
on utilisera simplement un éditeur de texte Emacs
(commande \texttt{emacs})\footnote{Attention à ne
  pas confondre éditeur de texte et traitement de texte.
  Le premier écrit du texte dans un format brut (ASCII, UTF-8\ldots)
  dans un fichier ; le second sert à \emph{mettre en page} du
  texte et écrit donc beaucoup d'autres informations que
  le texte dans les fichiers qu'il produit. Un code python mis en
  page avec libreOffice Writer ne pourra pas être interprété
  directement par Python.}.


Toute séquence de caractères sur une ligne
commençant par le caractère \verb+#+  est considérée comme un commentaire
et ne sera donc pas exécutée par l'interpréteur Python.

\begin{enumerate}
\item Dans le terminal, choisir le répertoire de travail approprié.
  Exécuter  le programme \texttt{script\_0.py} en  passant dans le
  terminal  la commande  \texttt{python3 script\_0.py}  ou la  commande
  \texttt{./script\_0.py}, comparer les résultats obtenus.\\
  NB : La commande \verb-chmod +x script_0.py- vous sera
  certainement utile. 

\item Ouvrir avec \texttt{emacs} le programme \texttt{script\_0.py}.

\item La première ligne de  votre script (qui est une version
  très simplifiée d'un programme) est ce que l'on nomme le \emph{shebang}.

\item Que  se passe-t-il après  avoir supprimé  la première  ligne du
  programme ?

\item Analyser la ligne 10 du programme \texttt{script\_0.py}.

\item Que se passe-t-il lorsque le code \texttt{$\backslash$n} est imprimé ?

\item Détailler les commandes en  charge de l'affichage des variables
  \texttt{a}, \texttt{b} et \texttt{c}.


\end{enumerate}
\begin{figure}  
  \lstinputlisting{../exemples/script_0.py}
  \caption{Contenu du fichier \texttt{script\_0.py}}
  \label{polynome_script_0}
\end{figure}





\subsection{Un peu d'interaction avec votre programme}




Nous  allons introduire  la  notion de  fonction
principale   \texttt{main()}  que   vous   trouverez  dans   l'exemple
\texttt{conditions.py}.


\begin{enumerate}

\item Ouvrir avec  \texttt{emacs} le  programme \texttt{conditions.py}
  et l'executer.
\item Ajouter des instructions \verb+print+
  pour afficher des messages de votre choix avant et après
  l'instruction \verb+main()+. Executer.
  En déduire l'ordre dans lequel sont executées les instructions de
  ce programme. 
\item Expliquer la fonctionnalité de la fonction \texttt{input()}.
\item De quels types sont les variables \texttt{name},
  \texttt{input\_number},
  et \texttt{number} ?
  Il est possible de demander à l'interpréteur Python cette
  information grâce à la fonction \texttt{type()}.


\item Le premier test conditionnel amorce le bloc d'instructions
  situé entre  les lignes 22  et 29.\\ Expliquer la  différence entre
  \texttt{if}, \texttt{elif} et \texttt{else}.

%2017-09-07 FM: !!! pas de or no not !!!
\item  Entre   les  lignes  32   et  39,  découvrez  le   mot-clef
  \texttt{and}. D'autres mots-clefs existent pour
  exprimer une condition  : \texttt{or}  et \texttt{not}.\\
  Proposer une manière de prendre en compte les cas non traité
  correctement jusqu'ici dans la dernière série de tests.
\end{enumerate}


\begin{figure}  
  \lstinputlisting{../exemples/conditions.py}
  \caption{Contenu du fichier \texttt{conditions.py}}
  \label{polynome_conditions}
\end{figure}



\subsection{Un peu de mathématiques}

\begin{enumerate}
\item Calculer la racine carrée de la valeur absolue de -1024.
  Comment faire en utilisant Python ?
  Il \textit{suffit} d'importer le module \texttt{numpy} et d'utiliser ensuite
  des fonctions définies dans ce module.
  
  Pour simplifier, un module est un bout de code que l'on trouve dans un fichier
  mettant à disposition un ensemble de fonctionnalités autour d'un thème commun.
  Pour connaître l'ensemble des possibilités du module \texttt{numpy}, il faut taper la commande
  \texttt{help("numpy")} dans l'interpréteur de commandes Python.

\item Ouvrir le programme \texttt{scientifique.py} et lire les commandes successives.
  La ligne 6 permet d'importer l'ensemble des fonctionnalités du module \texttt{numpy}.

\item Une force du langage Python est de disposer d'un nombre important de modules spécialisés.
  Découvrir par exemple les modules \texttt{time} ou \texttt{random} à l'aide de la fonction \texttt{help()}.
\end{enumerate}


\begin{figure}  
  \lstinputlisting{../exemples/scientifique.py}
  \caption{Contenu du fichier \texttt{scientifique.py}}
  \label{polynome_scientifique}
\end{figure}


\subsection{Manipuler les chaînes de caractères}


\begin{enumerate}

\item Editer le  fichier \texttt{chaine.py} puis,
  suite à l'analyse du programme, définir le type de chaque variable.

\item Expliquer ce que fait la ligne 20 du programme.

\item Comparer le nombre d'éléments constituant les variables
  \texttt{list\_of\_words} et  \texttt{sentence} ; sachant que la fonction \verb+len()+
  appliquée à une variable renvoie le nombre d'élément la constituant.
  
\item Détailler les fonctionalités des méthodes \texttt{split()}, \texttt{replace()} et \texttt{count()}.

\item Proposer une solution pour accéder au cinquième élément de la variable \texttt{list\_of\_words}.

\end{enumerate}


\begin{figure}  
  \lstinputlisting{../exemples/chaine.py}
  \caption{Contenu du fichier \texttt{chaine.py}}
  \label{polynome_chaine}
\end{figure}
